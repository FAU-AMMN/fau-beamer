% ..............................................................................
% Demo of the fau-beamer template.
%
% Copyright 2022 by Tim Roith <tim.roith@fau.de>
%
% This program can be redistributed and/or modified under the terms
% of the GNU Public License, version 2.
%
% ------------------------------------------------------------------------------
\documentclass[final]{beamer}

% ========================================================================================
% Theme: inner, outer, font and colors
% ----------------------------------------------------------------------------------------
\usepackage[institute=Nat,
			%ExtraLogo = template-art/DepartMath.pdf,
			%WordMark=None
			aspectratio=169,
			size=18
		   ]{styles/beamerthemefau}
% ----------------------------------------------------------------------------------------
% Input and output encoding
\usepackage[T1]{fontenc}
\usepackage[utf8]{inputenc}
% ----------------------------------------------------------------------------------------
% Language settings
\usepackage[english]{babel}


% ========================================================================================
% Fonts
% - Helvet is loaded by styles/beamerfonts
% - We use serif for math environements
% - isomath is used for upGreek letters
% ----------------------------------------------------------------------------------------
\usepackage{isomath}
\usefonttheme[onlymath]{serif}
\usepackage{exscale}
\usepackage{anyfontsize}
\setbeamercolor{alerted text}{fg=BaseColor}
% ----------------------------------------------------------------------------------------
% custom commands for symbols
\usepackage{styles/symbols}


% ========================================================================================
% Setup for Titlepage
% ----------------------------------------------------------------------------------------
\title[fau-beamer]{The NEW fau-beamer Template}
\subtitle{The \LaTeX{} template according to the 2021 FAU corporate guide}
\author[T. Roith]{Tim Roith}
\institute[FAU]{Friedrich-Alexander Universität Erlangen-Nürnberg, Department Mathematik}
\date{\today}


% ========================================================================================
% Bibliography
% ----------------------------------------------------------------------------------------
\usepackage{csquotes}
\usepackage[style=alphabetic, %alternatively: numeric, numeric-comp, and other from biblatex
			defernumbers=true,
			useprefix=true,%
			giveninits=true,%
			hyperref=true,%
			autocite=inline,%
			maxcitenames=5,%
			maxbibnames=20,%
			uniquename=init,%
			sortcites=true,% sort citations when multiple entries are passed to one cite command
			doi=true,%
			isbn=false,%
			url=false,%
			eprint=false,%
			backend=biber%
		   ]{biblatex}
\addbibresource{bibliography.bib}
\setbeamertemplate{bibliography item}[text]


% ========================================================================================
% Hyperref and setup
% ----------------------------------------------------------------------------------------
\usepackage{hyperref}
\hypersetup{
	colorlinks = true,
	final=true, 
	plainpages=false,
	pdfstartview=FitV,
	pdftoolbar=true,
	pdfmenubar=true,
	pdfencoding=auto,
	psdextra,
	bookmarksopen=true,
	bookmarksnumbered=true,
	breaklinks=true,
	linktocpage=true,
	urlcolor=BaseColor,
	citecolor=BaseColor,
	linkcolor=BaseColor
}


% ========================================================================================
% Additional packages
% ----------------------------------------------------------------------------------------



% ========================================================================================
% Various custom commands
% ----------------------------------------------------------------------------------------
\pdfsuppresswarningpagegroup=1 %solves the PDF inclusion problem
% Change color for cite locally
\newcommand{\colorcite}[3]{{\hypersetup{citecolor=#1}{\cite[#2]{#3}}}} 
% ----------------------------------------------------------------------------------------


% ========================================================================================
% The main document
% ----------------------------------------------------------------------------------------
\begin{document}
% Title page
\begin{trueplainframe}{}
\titlepage%
\end{trueplainframe}

% Introduction
\begin{frame}[t]{Introduction}{What is this?}
This file demonstrates the fau-beamer style, which is a style for the \LaTeX{} \texttt{beamer} class, which allows to create presentation slides in \LaTeX. The design is based on the FAU corporate \href{https://www.intern.fau.de/kommunikation-marketing-und-corporate-identity/corporate-identity/}{ style guide 2021}.

This code for this template was written by \href{https://timroith.github.io/}{Tim Roith}. If you have questions about the template or found a mistake you can send an email to tim\{dot\}roith\{at\}fau\{dot\}de or open a issue at the \href{https://github.com/FAU-AMMN/fau-beamer}{GitHub repository}.
\end{frame}

% Outline
\begin{frame}[t]{Outline}
\tableofcontents
\end{frame}

% Stylized outline
{
\setbeamertemplate{headline}[headline title]
\setbeamertemplate{sidebar left}[sidebar title theme]
\setbeamertemplate{frametitle}[frametitle title]
\setbeamertemplate{footline}[footline title]
\setbeamercolor{background canvas}{bg=BaseColor}
\setbeamercolor{section in toc}{fg=TitleFont}
\setbeamercolor{subsection in toc}{fg=TitleFont}

% toc frame
\begin{frame}{-}
\usebeamerfont{title}%
\hypersetup{linkcolor=TitleFont}
\tableofcontents
\end{frame}
}

% input exmple sections
% ------------------------------------------------------------------------------
\section{Example Section}
% ..............................................................................
\begin{frame}[t]{An Example Slide}{With a subtitle}
This is how a normal slide looks like, where plain text is put within the \texttt{frame} environment.
\end{frame}
%
%
%
%
%
% ..............................................................................
\begin{frame}[t]{Blocks}{}
You can use blocks with this template.
% ----------------
\begin{block}{Block headline}
This block spans all the way from the left to the right margin.
\end{block}
% ..............................................................................
Using minipages one can use blocks of certain fixed sizes.\\
%
\noindent%
\begin{minipage}[t]{0.45\textwidth}
\begin{block}{Headline A}
Some text.
\end{block}
\end{minipage}
\hfill%
% --------
\begin{minipage}[t]{0.2\textwidth}
\begin{block}{Headline B}
Some other text\\
that spans\\
over multiple lines.
\end{block}
\end{minipage}
\hfill%
% --------
\begin{minipage}[t]{0.2\textwidth}
\begin{block}{Headline C}
Even more text\\
that spans\\
over multiple lines.
\end{block}
\end{minipage}
%
\end{frame}
%
%
%
%
%
% ..............................................................................
\begin{frame}[t]{Top aligned}
In this frame everything is aligned on top.
\end{frame}
%
%
%
%
%
% ..............................................................................
\begin{frame}[b]{Bottom aligned}
In this frame everything is aligned on the bottom.
\end{frame}
%
%
%
%
%
% ..............................................................................
\begin{frame}{Lists}{Itemize and enumerate}
You can use the itemize environment that looks as follows.
% --------------------------------------
\begin{itemize}
\item An item.
\item Another one.
\begin{itemize}
\item A subitem.
\item Another subitem.
\end{itemize}
\item And another item.
\end{itemize}
%
\vfill%
You can also use the enumerate environment.
% --------------------------------------
\begin{enumerate}
\item The first item.
\item The second one.
\item The third one.
\end{enumerate}
% --------------------------------------
\end{frame}

% ..............................................................................
\begin{frame}{Using Overlay Specifications}
\begin{itemize}[<+(1)->]
\item The first item.
\item The second one.
\end{itemize}
\pause
\begin{block}{Framed Text}
This should be displayed after the list.
\pause
This should be displayed last.
\end{block}
\end{frame}
%
%
%
%
%
% ..............................................................................
\subsection{Subsection A}
\begin{frame}{\mbox{}}
We can also have subsections
\end{frame}
%
%
%
%
%
% ..............................................................................
\subsection{Subsection B}
\begin{frame}{\mbox{}}
When there is one subsection, there probably should be another.
\end{frame}
%
%
%
%
%
% ------------------------------------------------------------------------------
\section{The Color Scheme}
%
%
%
%
%
% ..............................................................................
\begin{frame}[fragile]{Choosing the Institution}
You can specify the institute template by passing it to package, i.e.,
%
\begin{center}
\begin{verbatim}
\usepackage[institute=<option>]{styles/beamerthemefau},
\end{verbatim}
\end{center}
%
where you have the following options,
\begin{itemize}
\item FAU
\item RW
\item Med
\item Nat
\item TF
\end{itemize}
\end{frame}
%
%
%
%
%
% ..............................................................................
\begin{frame}{The Colors}{Base and Dark scheme}
For each institute the color scheme consists of two main colors, which are named 
\texttt{BaseColor} and \texttt{BaseDarkColor}. They can be used throughout the document. For each of theses colors, adding a letter from A to D will create a lighter shade as displayed below.
%
%
%
%
%
\vspace{1em}%
\begin{columns}
\begin{column}{0.5\textwidth}
\setbeamercolor{colordemo}{bg=BaseColor}
\begin{beamercolorbox}[dp=0pt,sep=0em,wd=15cm,ht=10cm]{}
%
\begin{beamercolorbox}[sep=0em,wd=2cm,ht=2cm]{colordemo}%
\end{beamercolorbox}\quad\texttt{BaseColor}\newline%
\setbeamercolor{colordemo}{bg=BaseColorA}%
\begin{beamercolorbox}[sep=0em,wd=2cm,ht=2cm]{colordemo}
\end{beamercolorbox}\quad\texttt{BaseColorA}\hfill\newline%
\setbeamercolor{colordemo}{bg=BaseColorB}%
\begin{beamercolorbox}[sep=0em,wd=2cm,ht=2cm]{colordemo}%
\end{beamercolorbox}\quad\texttt{BaseColorB}\hfill\newline%
\setbeamercolor{colordemo}{bg=BaseColorC}%
\begin{beamercolorbox}[sep=0em,wd=2cm,ht=2cm]{colordemo}%
\end{beamercolorbox}\quad\texttt{BaseColorC}\hfill\newline%
\setbeamercolor{colordemo}{bg=BaseColorD}%
\begin{beamercolorbox}[sep=0em,wd=2cm,ht=2cm]{colordemo}%
\end{beamercolorbox}\quad\texttt{BaseColorD}%
\end{beamercolorbox}%
\end{column}
%
% 
%
\begin{column}{0.5\textwidth}
\setbeamercolor{colordemo}{bg=BaseDarkColor}
\begin{beamercolorbox}[dp=0pt,sep=0em,wd=15cm,ht=10cm]{}
%
\begin{beamercolorbox}[sep=0em,wd=2cm,ht=2cm]{colordemo}%
\end{beamercolorbox}\quad\texttt{BaseDarkColor}\newline%
\setbeamercolor{colordemo}{bg=BaseDarkColorA}%
\begin{beamercolorbox}[sep=0em,wd=2cm,ht=2cm]{colordemo}
\end{beamercolorbox}\quad\texttt{BaseDarkColorA}\hfill\newline%
\setbeamercolor{colordemo}{bg=BaseDarkColorB}%
\begin{beamercolorbox}[sep=0em,wd=2cm,ht=2cm]{colordemo}%
\end{beamercolorbox}\quad\texttt{BaseDarkColorB}\hfill\newline%
\setbeamercolor{colordemo}{bg=BaseDarkColorC}%
\begin{beamercolorbox}[sep=0em,wd=2cm,ht=2cm]{colordemo}%
\end{beamercolorbox}\quad\texttt{BaseDarkColorC}\hfill\newline%
\setbeamercolor{colordemo}{bg=BaseDarkColorD}%
\begin{beamercolorbox}[sep=0em,wd=2cm,ht=2cm]{colordemo}%
\end{beamercolorbox}\quad\texttt{BaseDarkColorD}%
\end{beamercolorbox}%
\end{column}
\end{columns}
\end{frame}
%
%
%
%
%
% ------------------------------------------------------------------------------
\section{The Frame Dimensions}
%
\setbeamercolor{colordemo}{fg=black,bg=BaseDarkColorC}
%
%
%
\begin{frame}[t]{Text Area}{The area you can fill in a normal frame}

\begin{beamercolorbox}[dp=0pt,sep=0em,wd=\textwidth,ht=\FrameHeight,center]{colordemo}%
This area can be filled on a normal frame.
\vspace{5cm}%
\end{beamercolorbox}%
\end{frame}
%
%
%
\begin{frame}

\begin{beamercolorbox}[dp=0pt,sep=0em,
					   wd=\textwidth,
					   ht=\dimexpr\FrameHeight+\TitleHeight+4mm,center]{colordemo}%
This area can be filled on a frame without a title.
\vspace{5cm}%
\end{beamercolorbox}%
\end{frame}
%
%
%
%
%
\begin{frame}[plain]
\begin{beamercolorbox}[dp=0pt,sep=0em,wd=\textwidth,ht=\paperheight,center]{colordemo}%
This area can be filled on a plain frame.
\vspace{5cm}%
\end{beamercolorbox}%
\end{frame}
%
%
%
%
%
\begin{trueplainframe}
\begin{beamercolorbox}[dp=0pt,sep=0em,wd=\textwidth,ht=\paperheight,center]{colordemo}%
This area can be filled on a true plain frame with removed margins.
\vspace{5cm}%
\end{beamercolorbox}%
\end{trueplainframe}
%
%
%
%
%
% ------------------------------------------------------------------------------
\section{Citing and bibliography}
%
%
%
%
%
\begin{frame}{Citing}
Since we use \texttt{biblatex} we can easily cite our favourite articles, like
\cite{bungert2021bregman, bungert2021neural}. For presentation the \texttt{footnote} command is useful to have citation at the bottom\footnote{\cite{bungert2021bregman}}.
\end{frame}
%
%
%
%
%
\begin{frame}{References}
To show the references used within this presentation we can use the \texttt{printbibliography} command from \texttt{biblatex}. For beamer documents it is important to give the additional option \texttt{[heading=none]}.
%
%
%
%
%
\begin{block}{Refernces}
\printbibliography[heading=none]
\end{block}
\end{frame}
%
%
%
%
%
\begin{frame}{fau-beamer}
%
\end{frame}
\end{document}